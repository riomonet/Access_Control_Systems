\documentclass[sigconf]{acmart}
\usepackage{paralist}

\acmConference[Conference]{Conference Name}{Date}{Location}
\acmYear{Year}

\begin{document}

\title{A Convenient and Secure Access Control of Physical Facilities using Smartphones}
\subtitle{Subtitle, if any}

\author{Author 1}
\affiliation{%
  \institution{CUNY Brooklyn College}
  \country{USA}
}
\email{author1@example.com}

\author{Author 2}
\affiliation{%
  \institution{CUNY Brooklyn College}
  \country{USA}
}
\email{author2@example.com}

\begin{abstract}
Your abstract goes here.
\end{abstract}

\keywords{Keyword1, Keyword2, Keyword3}

\maketitle

\section{Introduction}
Numerous facilities have controlled gate access. A common implementation is to
assign a user a physical token. Given the ubiquity of smartphones, we explore
the possibility of using a smartphone as a token. For instance, users
authenticates themselves via an application on their phones by entering a
password. We are concerned with two design requirements:
\begin{inparaenum}[1)] \item the system should be as conveniently to use as the
physical token;
    \item the authentication secrete, such as a password should not easily allow
shared access, just like the physical tokens. \end{inparaenum}

There are several proposals that allude to a possible solution to this problem,
particularly to satisfy the second requirement. These proposals including RFC
7800 are centered around the concept of ``Proof-of-Possession'' of an
authentication or authorization secrete~\cite{RFC7800,
I-D.ietf-oauth-pop-architecture, I-D.ietf-oauth-pop-key-distribution}.

In this report, we explore the design space for the intended application, and also 
provide a concise survey of ``Proof-of-Possession''. 



\section{Related Work}
Discuss related work here.

\section{Methodology}
Explain your methodology.

\section{Results}
Present your results.

\section{Conclusion}
Summarize your work and conclude.

\bibliographystyle{ACM-Reference-Format}
\bibliography{techreport}

\end{document}
